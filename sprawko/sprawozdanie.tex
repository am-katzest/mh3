\documentclass{article}
\usepackage[a4paper,margin=0.8in]{geometry}

\usepackage{amsmath} \usepackage{amssymb} \usepackage{amsfonts}
\usepackage{pifont}
\usepackage{enumitem}
\usepackage[T1]{fontenc}
\usepackage{lmodern}
\usepackage{rotating}
\usepackage{siunitx}
\usepackage[polish]{babel}
\usepackage[utf8]{inputenc}
\usepackage{multirow}
\usepackage[yyyymmdd]{datetime} \renewcommand{\dateseparator}{-} % ISO-8601
\usepackage{tabularx}
\usepackage{graphicx}
\usepackage{float}
\usepackage{hyperref}
\usepackage{minted}
\newcommand{\cljt}[1]{\mintinline{clojure}{#1}}
\newcommand{\inputgraph}[1]{

  \begin{figure}[H]
		\includegraphics[width=\linewidth]{#1}
  \end{figure}}
\begin{document}
\begin{minipage}{0.35\linewidth}
	\begin{tabular}{lr}
		Antoni Jończyk & 236551 \\
		Tomasz Roske   & 236639
	\end{tabular} \hfill
\end{minipage}
\hfill
\begin{minipage}{0.35\linewidth}
	\hfill Rok akademicki 2022/23 \par
	\hfill czwartek, 13:00
\end{minipage}
\bigskip \bigskip \bigskip \bigskip \bigskip
\begin{center}
	\textbf{Metaheurystyki i ich zastosowania, zadanie 3}\\
	\bigskip
	\large implementacja algorytmu mrówkowego
\end{center}
\bigskip \bigskip
\section{Działanie programu}

\subsection{struktury danych}
\indent
atrakcja:
\inputminted{fsharp}{snippets/Program.fsx_atrakcja}
mrówkę w trakcie jej ruchu opisujemy w następujący sposób:
\inputminted{fsharp}{snippets/Program.fsx_antr}
a po jego zakończeniu statystyki są przechowywane w:
\inputminted{fsharp}{snippets/Program.fsx_ants}
dane dotyczące krawędzi przechowujmy w tablicach jednowymiarowych, indeksowanych w następujący sposób:
\inputminted{fsharp}{snippets/Program.fsx_index}

\subsection{mrówki}
dla każdego pokolenia wypuszczamy zadaną liczbę mrówek
\inputminted{fsharp}{snippets/Program.fsx_advance}
każda odwiedza wszystkie atrakcje, w każdym kroku wybierając następną korzystając z funkcji:
\inputminted{fsharp}{snippets/Program.fsx_dir}
dla losowania ruletką szansa na wybranie danej atrakcji zależy od:
\inputminted{fsharp}{snippets/Program.fsx_liek}
następnie sumujemy ich ślady z wagą zależną od długości przebytej trasy i
dodajemy do odpowiednio osłabionego śladu z poprzedniego stanu
\inputminted{fsharp}{snippets/Program.fsx_trails}
\section{analiza wyników}
\subsection{wielkość populacji}
Na grafach możemy zobaczyć, że większa populacjia zwykle oznacza szybsze
znajdywanie minimum, jednak wyższy koszt obliczeniowy nie czyni zwiększania tego
parametru zbyt korzystnym. Ze względu na wysoki koszt obliczeniowy
postanowiliśmy przeprowadzić mniej symulacji niż dla innych parametrów, stąd na
niektórych grafach lepsze wyniki uzyskały próby z mniejszą populacją.
\inputgraph{1l.png}
\inputgraph{2l.png}
\inputgraph{3l.png}
\inputgraph{4l.png}
\inputgraph{5l.png}
\inputgraph{6l.png}

\subsection{parametry $\alpha$ i $\beta$}
$\alpha=3$ we wszystkich próbach okazała się lepsza niż $\alpha=1$ a wartości $\beta=2$ dają
lepsze wyniki w większości prób niż $\beta=1$
\inputgraph{1w.png}
\inputgraph{2w.png}
\inputgraph{3w.png}
\inputgraph{4w.png}
\inputgraph{5w.png}
\inputgraph{6w.png}
\subsection{wpływ szansy na wybór losowy}
dla każdego ruchu mrówka ma pewną szansę na zignorowanie feromonów i
heurystyki, a wybrania zupełnie losowej ścieżki, na poniższych wykresach
widzimy, że 0.3 jest dla większości zestawów danych zdecydowanie zbyt dużą
wartością, sprawdziliśmy również 0.1 i 0.02, które sprawdzają się lepiej dla
różnych map, dla tych większych lepsze wyniki zapewnia mniejsza wartość.
\inputgraph{1s.png}
\inputgraph{2s.png}
\inputgraph{3s.png}
\inputgraph{4s.png}
\inputgraph{5s.png}
\inputgraph{6s.png}
\subsection{szybkość parowania}
wyniki różnią się, jednak nie w żaden łatwy do określenia sposób
\inputgraph{1p.png}
\inputgraph{2p.png}
\inputgraph{3p.png}
\inputgraph{4p.png}
\inputgraph{5p.png}
\inputgraph{6p.png}
\end{document}
