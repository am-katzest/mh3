\documentclass{article}
\usepackage[a4paper,margin=0.8in]{geometry}

\usepackage{amsmath} \usepackage{amssymb} \usepackage{amsfonts}
\usepackage{pifont}
\usepackage{enumitem}
\usepackage[T1]{fontenc}
\usepackage{lmodern}
\usepackage{rotating}
\usepackage{siunitx}
\usepackage[polish]{babel}
\usepackage[utf8]{inputenc}
\usepackage{multirow}
\usepackage[yyyymmdd]{datetime} \renewcommand{\dateseparator}{-} % ISO-8601
\usepackage{tabularx}
\usepackage{graphicx}
\usepackage{float}
\usepackage{hyperref}
\usepackage{minted}
\newcommand{\cljt}[1]{\mintinline{clojure}{#1}}
\newcommand{\inputgraph}[1]{\newpage \input{#1}\newpage} % p, n w przeglądarce ^w^
\begin{document}
\begin{minipage}{0.35\linewidth}
	\begin{tabular}{lr}
		Antoni Jończyk & 236551 \\
		Tomasz Roske   & 236639
	\end{tabular} \hfill
\end{minipage}
\hfill
\begin{minipage}{0.35\linewidth}
	\hfill Rok akademicki 2022/23 \par
	\hfill czwartek, 13:00
\end{minipage}
\bigskip \bigskip \bigskip \bigskip \bigskip
\begin{center}
	\textbf{Metaheurystyki i ich zastosowania, zadanie 3}\\
	\bigskip
	\large implementacja algorytmu mrówkowego
\end{center}
\bigskip \bigskip
\section{Działanie programu}

\subsection{struktury danych}
\indent
atrakcja:
\inputminted{fsharp}{snippets/Program.fsx_atrakcja}
mrówkę w trakcie jej ruchu opisujemy w następujący sposób:
\inputminted{fsharp}{snippets/Program.fsx_antr}
a po jego zakończeniu statystyki są przechowywane w:
\inputminted{fsharp}{snippets/Program.fsx_ants}
dane dotyczące krawędzi przechowujmy w tablicach jednowymiarowych, indeksowanych w następujący sposób:
\inputminted{fsharp}{snippets/Program.fsx_index}

\subsection{ruch mrówek}
funkcja atrakcyjności dla mrówki danej krawędzi:
\inputminted{fsharp}{snippets/Program.fsx_liek}
funkcja wybierająca kierunek następnego kroku mrówki:
\inputminted{fsharp}{snippets/Program.fsx_dir}

\subsection{symulacja}
dla każdego kroku symulacji wykonywane są następujące czynności:
\inputminted{fsharp}{snippets/Program.fsx_advance}
sumowanie śladów:
\inputminted{fsharp}{snippets/Program.fsx_trails}


\section{analiza wyników}
\subsection{co to jest i jak to czytać}
\subsection{prawdopodobieństwo mutacji}
ten parametr decyduje o względnych szansach mutacji i krzyżowania
\\
\end{document}
